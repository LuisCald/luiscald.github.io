\documentclass{resume}[10pt] % Use the custom resume.cls style
\usepackage{changepage}
\usepackage{graphicx}
\usepackage{multicol}
\setlength{\columnsep}{2cm}
%    PACKAGES AND OTHER DOCUMENT CONFIGURATIONS
\usepackage{xcolor}
\usepackage[colorlinks = true,
            linkcolor = blue,
            urlcolor  = blue,
            citecolor = blue,
            anchorcolor = blue]{hyperref}
\usepackage{fdsymbol}

\newcommand{\MYhref}[3][blue]{\href{#2}{\color{#1}{#3}}}
\usepackage[left=0.5in,top=0.5in,right=0.5in,bottom=0.5in]{geometry} % Document margins
\newcommand{\tab}[1]{\hspace{.2667\textwidth}\rlap{#1}}
\newcommand{\itab}[1]{\hspace{0em}\rlap{#1}}
\newenvironment{subs}
  {\adjustwidth{3em}{0pt}}
  {\endadjustwidth}
\name{Luis Calderon} % Your name
% \address{Heerstrasse 51, 53111 Bonn, Germany}
\address{Kaiserplatz 7-9, 53113 Bonn, 4th floor \\ luis.calderon@uni-bonn.de  \\ github: \MYhref[purple]{https://github.com/LuisCald}{\textbf{@LuisCald}} \\ linkedin: \MYhref[purple]{https://www.linkedin.com/in/luiscald}{\textbf{@luiscald}} }
\begin{document}

%----------------------------------------------------------------------------------------
%    EDUCATION SECTION
%----------------------------------------------------------------------------------------
\footnotesize
\begin{rSection}{Education}
\\{\bf University of Bonn, Germany} \hfill {\em October 2020 - Present}
\\ Ph.D. in Economics\\
M.Sc. in Economics with a specialization in research \\

\\{\bf University of Bonn, Germany} \hfill {\em October 2018 - October 2020}
\\ M.Sc. in Economics\\

{\bf Florida International University, Miami, FL} \hfill {\em August 2013 - December 2017}
\\ B.B.A. in Finance
\\ B.A. in Economics, \texit{Minor in Mathematics}
\end{rSection}
%------------------------------------------------------------------
%    TECHNICAL STRENGTHS SECTION
%------------------------------------------------------------------
\footnotesize
\begin{rSection}{Research Experience}
\\{\bf Institute of Macroeconomics and Econometrics, University of Bonn}\hfill {\em June 2020 - Present}\\
\textsc{PIs}: Christian Bayer, Benjamin Born, Ralph Luetticke, and Moritz Kuhn\\
\textsc{Project 1}: \href{https://www.ralphluetticke.com/files/BBL_Inequality_Sep2022.pdf}{Shocks, Frictions, and Inequality in US Business Cycles} \\
% \textsc{Tools}: Stata, Python, and MATLAB \\
\textsc{Master Thesis}: \href{https://github.com/LuisCald/Master-Thesis/blob/main/master_thesis.pdf}{The Effects of Income Uncertainty on Liquidity, Default, and Prosperity} \\
\textsc{Project 2}: \href{https://github.com/LuisCald/DistributionalDynamics}{Distributional Dynamics} \\
\textsc{Project 3}: \href{https://github.com/BASEforHANK}{BASEforHANK} \\
\textsc{Project 4}: {Job Levels II \\

% \textsc{Admin Support:} For RTG progress report and proposal for renewal\\
% \text{\hspace{32mm}}Organize MEF Seminar Series\\


{\bf Institute on Behavior \& Inequality (Briq)} \hfill {\em May 2019 - June 2020}\\
\textsc{Project 1}: \href{https://drive.google.com/file/d/1sQmazQaMzSg6RY7dHfJwmApYOkhFqtrp/view}{Bureaucracy as a Tool for Politicians}\\
\textsc{Project 2}: \href{https://drive.google.com/file/d/1TzPZ8FT4NvlW-lxk3L13H4BTwC0Jkd4U/view?usp=sharing}{Spatial standard errors for several commonly used M-estimators}\\
% \textsc{Econometric Methods}: RDD \\
% \textsc{Tools}: Stata and Python \\
\textsc{Under}: \href{http://www.leanderheldring.com/}{Prof. Leander Heldring}
\end{rSection}

\footnotesize
\begin{rSection}{Teaching Experience}
{\bf Bonn Graduate School of Economics}\\
\textsc{Class}: Coding in Julia and Stata (PhD) \hfill {\em Winter 2021 - Present}\\
\textsc{Class}: Topics in Inequality (BSc)\hfill {\em Summer 2024}\\
\textsc{Class}: Macroeconomics I (BSc)\hfill {\em Summer 2025, Summer 2026}\\
\textsc{Winter School}: BASEforHANK (for PhDs and central bankers) \hfill {\em Dec 2025}
\end{rSection}

\footnotesize
\begin{rSection}{Working Papers}
% \textbf{``Spatial Standard Errors for Several Commonly used M-estimators"}\textit{- with Leander Heldring} \\(\textsc{Stata/Python Package Implementation})\\
% \\
\textbf{``Distributional Dynamics"} \textit{- with Christian Bayer and Moritz Kuhn} --- submitted \\
\textbf{``Spatial Standard Errors for Several Commonly used M-estimators"}\textit{- with Leander Heldring}\\ \textit{R\&R} at International Journal of Computational Economics and Econometrics\\
\end{rSection}

\begin{rSection}{Works in Progress}
% \\
\textbf{``Distributional Counterfactuals"} \\
\textbf{``Monetary Policy According to Data"}
\end{rSection}
% \textbf{``Consumption Smoothing and Credit Access Across Two Financial Regimes"} \\(\textsc{Idea})\\\\
% \textbf{``Bubbles, Balloons and Banks: Economic Stability and Asset Price Transmission"} \\(\textsc{Early Stage})\\\\

\footnotesize
\begin{rSection}{Academic Events}
\textbf{Summer Schools:} Macroeconomics and Financial Markets (June 2022, Dean Corbae), DSE 2024  \\
\textbf{Spring Schools:} Macroeconomics and Big Data (April 2024, Domenico Giannone, Giorgio Primiceri) \\
\textbf{Recent Conferences:} CASFI Summer 2025, BCFM 2025, CRC Retreat 2025, Summer SED 2024 (Barcelona, Presenter), Banca d'Italia 2024 (Rome, Poster), Firm Heterogeneity and Macroeconomics (Bonn/Mannheim, 2022-2024), Center for Advance Studies in Finance and Inequality (Bonn, 2024), Money Conference (Board of Governors, 2024), HANK Conference (Berlin, 2023)  \\
\textbf{Workshops:} Empirical Methods in Macroeconomics (May 2024, Christian Wolf)
\end{rSection}
\newpage
% \begin{rSection}{Additional Information \& Future Plans}

% \item \textbf{Technical Strengths}\\\\
% \renewcommand*{\arraystretch}{1.5}
%     \begin{tabular}{|c|c|}
%     \hline
%         Language & Years of Exp.\\\hline
%         \includegraphics[height=0.8em]{stata_logo.png} & 5 years  \\
%         \includegraphics[height=0.9em]{python_logo.png} & 3 years \\
%         \includegraphics[height=0.9em]{julia_logo.png} & 3 months \\
%         MATLAB & 2 years \\
%         R & \textsc{For ggplot2}\\ \hline
%     \end{tabular}

% \item \textbf{\textsc{Interests}}: Methods in Macro, Fiscal Policy, Portfolio Choice, Banking \& Financial Regulation

% \textbf{Past Semester Coursework} \begin{itemize}
% \itemsep -4pt
%     \item Topics in Portfolio Choice
%     \item Topics in Macroprudential Policies
%     \item Topics in Empirical Banking
%     \item Certification in Machine Learning (in progress)
% \end{itemize}

% \item \textbf{Short-term Plans}\begin{itemize}
%     \item Transform the \textbf{Distributional Dynamics} paper to a working paper
%     \item Push aggregate uncertainty channel in \textbf{Multipliers} paper and make an expose on the mechanisms at play when comparing their multiplier effects
% \end{itemize}


% \end{rSection}

 %      Leander Heldring, Assistant Professor  \\
 % Kellogg School of Management, Northwestern University \\
 %  leander.heldring@kellogg.northwestern.edu \\\\

\footnotesize
\begin{rSection}{References}
\setlength{\columnsep}{25pt}
\begin{multicols}{3}
Christian Bayer, Prof. of Economics\\
 University of Bonn \\
  Christian.Bayer@uni-bonn.de \\\\
  Moritz Kuhn, Prof. of Economics\\
 University of Mannheim \\
  mokuhn@uni-mannheim.de \\\\
  Farzad Saidi, Prof. of Economics\\
 University of Bonn \\
  saidi@uni-bonn.de
\end{multicols}
\end{rSection}

\footnotesize
\begin{rSection}{Membership and Roles}
2024 - present: Member of Collaborative Research Center Transregio 224 "Economic Perspectives on Societal Challenges" \\
2024 - present: Kreisliga Basketball (PG), Volunteer for U18 games\\
2022 - present: Bonner Liga Basketball (PG and Treasurer)\\
2021 - 2024: BGSE Representative \\
2021 - present: NRW Liga Chess Player
\end{rSection}
\begin{rSection}{Supervision of Master Thesis}
2025: Marius Trovik (1.0)\\
2024: Zsigmond Szajbely (1.0) \\
2023: Luke Liscio (1.0)\\
2022: Islom Juraev (2.3)\\
2021: Anna Sanchez (1.7)
\end{rSection}

\begin{rSection}{Other}
\textbf{Citizenship}: American \\
\textbf{Background}: Father: Peruvian, Mother: Nicaraguan \\
\textbf{Languages}: English (Native), Spanish (Native), German (B1), Japanese (A1) \\ \textbf{Hobbies}: Basketball, Chess, Coding, Cooking, Sketching
\end{rSection}
\end{document}
